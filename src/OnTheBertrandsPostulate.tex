\documentclass[12pt,letterpaper,oneside,reqno]{amsart}
\usepackage{amsfonts}
\usepackage{amsmath}
\usepackage{amssymb}
\usepackage{amsthm}
\usepackage{float}
\usepackage[font=small,labelfont=bf]{caption}
\usepackage[left=1in,right=1in,bottom=1in,top=1in]{geometry}
\usepackage[pdfpagelabels,hyperindex,colorlinks=true,linkcolor=blue,urlcolor=magenta,citecolor=green]{hyperref}

\newcommand \coeffA [3][A] {{\mathbf{#1}} \sb{#2,#3}}

\newtheorem{thm}{Theorem}[section]
\newtheorem{lem}{Lemma}[section]
\newtheorem{conj}[thm]{Conjecture}

%--------Meta Data: Fill in your info------
\title[Goldbach-like conjectures via Bertrand's Postulate]{Goldbach-like conjectures via Bertrand's Postulate}
\author[Petro Kolosov]{Petro Kolosov}
\email{kolosovp94@gmail.com}
\keywords{Bertrand's Postulate, Goldbach Conjecture, Ternary Goldbach Conjecture}
\urladdr{https://kolosovpetro.github.io}
\subjclass[2010]{11P32, 11A41, 97F60}
\date{\today}
\hypersetup{
    pdftitle={Goldbach-like conjectures via Bertrand's Postulate},
    pdfsubject={Discrete Mathematics, Number Theory},
    pdfauthor={Petro Kolosov},
    pdfkeywords={Bertrand's Postulate, Goldbach Conjecture, Ternary Goldbach Conjecture}
}
\begin{document}
    \begin{abstract}
        In this manuscript the ternary and binary Goldbach's conjectures are reviewed from Bertrand's postulate prospective.
        As a result, a relations between both ternary and binary Goldbach's conjectures and Bertrand's postulate
        are established.
        Furthermore, a three Goldbach-like conjectures are proposed and discussed.
        Verification programs are attached to the last section.
    \end{abstract}
    \maketitle
    \tableofcontents


    \section{Introduction} \label{sec:introduction}
    In number theory, Bertrand's postulate is a statement that was first conjured in 1845 by
    Joseph Bertrand~\cite{bertrand1845}.
    \begin{thm}
        \label{bertrand_theorem} (Bertrand–Chebyshev theorem.)
        For every positive integer $n>1$ exists at least one prime $p$ such that
        \[
            n < p < 2n
        \]
    \end{thm}
    Bertrand's postulate completely proved by Chebyshev in 1852~\cite{Tchebichef1852}.
    From Bertrand's postulate immediately follows
    \begin{lem}
        \label{bertrands_partition_lemma} (Even Bertrand's partition.)
        By Bertrand's postulate, for every positive integer $n>1$ there is at least one partition such that
        \[
            2n = p + \verb!odd!,
        \]
        where \verb!odd! is odd member of Even Bertrand's partition.
    \end{lem}
    Now we have to recall ternary Goldbach's conjecture or namely Goldbach's weak conjecture
    \begin{conj}
        \label{ternary_goldbach_conjecture} (Ternary Goldbach's Conjecture.)
        Every odd number greater than 7 can be expressed as the sum of three odd primes.
        \[
            2n+1 = \mathfrak{p}_i +  \mathfrak{p}_j + \mathfrak{p}_k, \quad n > 2,
        \]
        where $\mathfrak{p}_i +  \mathfrak{p}_j + \mathfrak{p}_k$ is ternary Goldbach partition.
    \end{conj}
    Ternary Goldbach's conjecture is claimed to be true by H.A Helfgott~\cite{helfgott2013minor, helfgott2014ternary}.
    Furthermore, the proof was clarified in~\cite{helfgott2015ternary}.
    From this prospective, let express the lemma~\ref{bertrands_partition_lemma} for positive odd numbers,
    \begin{lem}(Odd Bertrand's partition.)
        \label{bertrands_odd_partition_lemma}
        By Bertrand's postulate, for every positive integer $n>1$ there is at least one partition such that
        \[
            2n + 1 = p + \verb!odd! + 1 = p + \verb!even!,
        \]
        where \verb!even! is even member of Odd Bertrand's partition.
    \end{lem}
    Then we have the following relation between Odd Bertrand's Partition~\ref{bertrands_odd_partition_lemma} and
    Ternary Goldbach's Partition~\ref{ternary_goldbach_conjecture}, for every positive integer $n>2$
    \begin{equation}
        2n + 1 = p + \verb!odd! + 1 = \mathfrak{p}_i +  \mathfrak{p}_j + \mathfrak{p}_k
        \label{eq:goldbach_and_bertrand_relation}
    \end{equation}
    From equation~\eqref{eq:goldbach_and_bertrand_relation} imply the Goldbach-like conjectures
    \begin{conj}
        For every prime $p< 2n-1$ in positive Odd $2n+1, \; n>2$ Bertrand's Partition~\ref{bertrands_odd_partition_lemma},
        the prime $p$ is always a member of Ternary Goldbach's Partition
        \[
            2n+1 = \mathfrak{p}_i +  \mathfrak{p}_j + p
        \]
    \end{conj}
    \begin{conj}
        For every positive odd $2n+1, \; n>2$ Bertrand's partition~\ref{bertrands_odd_partition_lemma},
        the prime $p < 2n-1$ always satisfies
        \[
            \bigvee_{t \in \{i, j, k\}} \mathfrak{p}_t = p,
        \]
        where $\mathfrak{p}_i, \; \mathfrak{p}_j, \; \mathfrak{p}_k$ are members of ternary Goldbach's partition of $2n+1$.
    \end{conj}
    \begin{conj}
        For every positive odd $2n+1, \; n>2$ Bertrand's partition~\ref{bertrands_odd_partition_lemma}
        the even part is always a sum of two primes
        \[
            2n + 1 = p + \verb!even! \rightarrow \verb!even! = \mathfrak{p}_j + \mathfrak{p}_k, \; n > 2
        \]
    \end{conj}


    \section{Discussion}\label{sec:discussion}
    Consider the Conjecture 1.3.
    Suppose it is true.
    Then for every even integer $2k>2$, take an odd prime $p<2k$, and let $n=k+\frac{p-1}{2}$.
    Then $n<p<2n-1$, so the true scenario would imply that there is a ternary Goldbach partition of $2n+1$ containing $p$.
    But $2n+1=2k+p$, giving us a binary Goldbach partition of $2k$.

    Suppose the Conjecture 1.3 is false.
    Then for some $n > 2$, there is a prime $p$ such that $n<p<2n-1$ and $p$ is not a member
    of an odd Goldbach partition of $2n+1$.
    It implies that $2n+1-p$ is an even integer which is not the sum of two primes.

    The Conjecture 1.3 implies Goldbach strong conjecture, however from different prospective.
    It doesn't assume that all positive even numbers greater than 4 are sum of two primes,
    but only provides a relation between Bertrand's postulate and ternary Goldbach partition.
    Conjecture 1.3 is immediately true if Goldbach's strong conjecture is true.
    Conjectures 1.4, 1.5 are following directly from Conjecture 1.3.


    \section{Verification}\label{sec:verification}
    Conjecture 1.3 may be verified up to $5 \times 10^3$ via the program~\cite{kolosov2021github}.
    However, in order to verify larger bounds, the program should be optimized and rewritten using any
    low-level programming language, for instance, C or C++.
    Currently, it has an asymptotic complexity of $O(n^2)$ and written on high-level language C\#.
    The verification results are at
    \href{https://github.com/kolosovpetro/GoldbachConjecture/blob/master/GoldbachConjecture.BertrandValues.UI/PartitionsTo5000.txt}
    {\texttt{github.com/kolosovpetro/GoldbachConjecture/PartitionsTo5000.txt}}
    \bibliographystyle{unsrt}
    \bibliography{OnTheBertrandsPostulate}
\end{document}